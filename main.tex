\documentclass[12pt,oneside]{article}
\usepackage{enumerate}
\usepackage{fancyhdr}
\usepackage{a4wide}
\usepackage{titlesec}
\usepackage{enumitem}
\usepackage[utf8]{inputenc}
\usepackage{graphicx} % Required for inserting images

\usepackage[table]{xcolor}


\setlength{\arrayrulewidth}{0.5mm}
\setlength{\tabcolsep}{10pt}
\renewcommand{\arraystretch}{2.5}

\definecolor{lightblue}{HTML}{b0c4de}
\definecolor{lightsteelblue}{HTML}{add8e6}

\usepackage{hyperref}

%%%%%%%%%%%%%%%%%%%%%%%%%%%%%%%%


\begin{document}

%pagina introduttiva

\begin{titlepage}
    \begin{flushright}
        \textbf{Corso di Fondamenti di Intelligenza Artificiale}
        \textbf{\\Università degli Studi di Salerno}
    \end{flushright}
    \vspace*{1.5cm}
    \centering
    \includegraphics[width=0.4\textwidth]{logoUNISA.png}
    \vfill
    \Huge\textbf{BOOKS}
    \vspace{1ex}
    \rule{\linewidth}{1pt}
    \Large\textbf{Maria Angela Mancuso \\
        Ines Malfettone \\
        Federico Santonicola \\
        Attilio Sessa}
    \vfill
    \today
\end{titlepage}

%indice
\clearpage %crea nuova pagina

\setcounter{page}{1}

\begin{flushright}
        \Large\textbf{Indice}
\end{flushright}
\rule{\linewidth}{1pt}



%1
\clearpage
\setcounter{section}{0}
\section{Introduzione}
    \begin{enumerate}
        \item Scopo del progetto
    \end{enumerate}
    \begin{flushleft}
    
        Il nostro progetto si pone come obiettivo lo studio e la sperimentazione di differenti tecniche di Machine Learning capaci di analizzare ed estrarre informazioni da dati sotto forma di linguaggio naturale. Nello specifico si è interessati alla categorizzazione di libri tramite una breve descrizione testuale ed un elenco di autori. La categorizzazione è stata elaborata tramite due tecniche di Machine Learning, Classificazione e Clustering, che, seppur facenti utilizzo di due approcci diversi (rispettivamente apprendimento supervisionato e apprendimento non supervisionato), in linea teorica dovrebbero essere in grado di riportare risultati similari e confrontabili. A tal proposito occorrerà addestrare più modelli differenti per entrambe le tecniche e verificare quali di questi permette di ottenere il miglior risultato.
        
    \end{flushleft}
\section{Specifiche del progetto}
    \begin{enumerate}
        \item Ambiente: PEAS
    \end{enumerate}
    \begin{flushleft}
        Specifica PEAS
    \end{flushleft}
   
%tabella
%\href{https://www.overleaf.com/}{Specifiche del progetto}
    \begin{table}[ht]
    \centering
    \rowcolors{2}{lightblue}{lightsteelblue}
    \begin{tabular}{ | p{3cm} | p{9cm} | }
    \hline
    \multicolumn{2}{|c|}{PEAS} \\
    \hline
    Performance & La misura di prestazione è la capacità di avvicinarsi quanto più possibile al corretto genere del libro in questione. È necessario utilizzare misure di prestazione differenti per valutare Classificazione e Clustering. Nel caso della Classificazione si è usato: Accuratezza, Report di classificazione, che comprendono Precisione, Recall e F1-score per ciascuna classe, e una Matrice di  confusione. Per il clustering, invece,  si è utilizzato il Silhouette Score.\\
    \hline
    Environment & I nostri modelli sono stati realizzati e operano nell'ambiente di sviluppo PyCharm il quale presenta le seguenti caratteristiche: \begin{itemize}
        \item completamente osservabile: il modello ha la visione completa del dataset e degli attributi associati a ciascun libro.
        \item deterministico: una volta addestrato un modello, la variazione dello stato dell'ambiente rimane la stessa a fronte degli stessi input.
        \item episodico: l'agente delibera a fronte di determinati episodi che consistono in nuove richieste di predizione.
        \item statico: l'ambiente resta invariato mentre l'agente opera.
        \item discreto: viene fornito un insieme discreto di informazioni per ciascun libro.
        \item singolo: l'ambiente permette di addestrare più modelli ma questi vengono valutati singolarmente.
    \end{itemize}
    
    -completamente osservabile: il modello ha la visione completa del dataset e degli attributi associati a ciascun libro.
    -deterministico: una volta addestrato un modello, la variazione dello stato dell'ambiente rimane la stessa a fronte degli stessi input.
    -episodico: l'agente delibera a fronte di determinati episodi che consistono in nuove richieste di predizione.
    -statico: l'ambiente resta invariato mentre l'agente opera.
    -discreto: viene fornito un insieme discreto di informazioni per ciascun libro.
    -singolo: l'ambiente permette di addestrare più modelli ma questi vengono valutati singolarmente.
    
    \hline
    Actuators& Gli agenti mostrano i risultati attraverso due tipi di attuatori:  %elenco
    -console dell'ambiente di sviluppo: durante l'addestramento e testing i modelli riportano informazioni di controllo e i risultati ottenuti
    -grafici esplicativi: mostrano informazioni di vario tipo, tra cui analisi del dataset e risultati ottenuti dalle predizioni dei modelli.\\
    \hline
    Sensors& I modelli ricevono le informazioni necessarie per l'addestramento tramite un file contenente il dataset. Inoltre è possibile specificare tramite console nuovi dati su cui effetturare nuove predizioni. \\
    \hline
    \end{tabular}

\caption{Tabella della specifica PEAS}
\label{table:ta}

\end{table}

    
\end{document}
